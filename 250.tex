
\documentclass{article}
%%%%%%%%%%%%%%%%%%%%%%%%%%%%%%%%%%%%%%%%%%%%%%%%%%%%%%%%%%%%%%%%%%%%%%%%%%%%%%%%%%%%%%%%%%%%%%%%%%%%%%%%%%%%%%%%%%%%%%%%%%%%%%%%%%%%%%%%%%%%%%%%%%%%%%%%%%%%%%%%%%%%%%%%%%%%%%%%%%%%%%%%%%%%%%%%%%%%%%%%%%%%%%%%%%%%%%%%%%%%%%%%%%%%%%%%%%%%%%%%%%%%%%%%%%%%
%TCIDATA{OutputFilter=LATEX.DLL}
%TCIDATA{Version=5.50.0.2960}
%TCIDATA{<META NAME="SaveForMode" CONTENT="1">}
%TCIDATA{BibliographyScheme=Manual}
%TCIDATA{Created=Friday, May 10, 2024 15:35:39}
%TCIDATA{LastRevised=Wednesday, May 15, 2024 11:03:43}
%TCIDATA{<META NAME="GraphicsSave" CONTENT="32">}
%TCIDATA{<META NAME="DocumentShell" CONTENT="Standard LaTeX\Blank - Standard LaTeX Article">}
%TCIDATA{CSTFile=40 LaTeX article.cst}

\newtheorem{theorem}{Theorem}
\newtheorem{acknowledgement}[theorem]{Acknowledgement}
\newtheorem{algorithm}[theorem]{Algorithm}
\newtheorem{axiom}[theorem]{Axiom}
\newtheorem{case}[theorem]{Case}
\newtheorem{claim}[theorem]{Claim}
\newtheorem{conclusion}[theorem]{Conclusion}
\newtheorem{condition}[theorem]{Condition}
\newtheorem{conjecture}[theorem]{Conjecture}
\newtheorem{corollary}[theorem]{Corollary}
\newtheorem{criterion}[theorem]{Criterion}
\newtheorem{definition}[theorem]{Definition}
\newtheorem{example}[theorem]{Example}
\newtheorem{exercise}[theorem]{Exercise}
\newtheorem{lemma}[theorem]{Lemma}
\newtheorem{notation}[theorem]{Notation}
\newtheorem{problem}[theorem]{Problem}
\newtheorem{proposition}[theorem]{Proposition}
\newtheorem{remark}[theorem]{Remark}
\newtheorem{solution}[theorem]{Solution}
\newtheorem{summary}[theorem]{Summary}
\newenvironment{proof}[1][Proof]{\noindent\textbf{#1.} }{\ \rule{0.5em}{0.5em}}
\input{tcilatex}
\begin{document}

\tableofcontents

\section{Computer Architecture and Organization}

\subsection{Basics:}

\subsubsection{Computer Architecture}

Deals with the\textbf{\ functional behaviou}r of Computer Systems.

\textbf{Design Implementation }for the various parts of the computer

\textit{(Designing)}

\subsubsection{Computer Organization}

Deals with \textbf{structural relationship}

Operational attributes are linked together and contribute to realize the
architectural specification

\textit{(Utilization)}

\bigskip

\textbf{Computer organization} refers to the operational units and their
interconnections that realize the architectural specifications.

\bigskip . Examples of architectural attributes include the instruction set, 
\textbf{the number of bits} used to represent various data types (e.g., 
\textbf{numbers, characters}),\textbf{\ I/O mechanisms}, and \textbf{%
techniques for addressing memory}. \bigskip\ Organizational attributes
include those hardware details transparent to the programmer, such as\textbf{%
\ control signals}; \textbf{interfaces between the computer and peripherals}%
; and the \textbf{memory technology used}.

\bigskip

\subsubsection{Functional Units}

Processor: We can think of this as the brain of the system.

\[
\FRAME{itbpF}{2.4396in}{1.8481in}{0in}{}{}{Figure}{\special{language
"Scientific Word";type "GRAPHIC";display "USEDEF";valid_file "T";width
2.4396in;height 1.8481in;depth 0in;original-width 6.6374in;original-height
5.4613in;cropleft "0";croptop "1";cropright "1";cropbottom "0";tempfilename
'SDAC2G01.bmp';tempfile-properties "XPR";}} 
\]

\bigskip

Memory: Stores all the instructions for the processor to work accordingly.

It can also store data

\[
\FRAME{itbpF}{2.162in}{1.5056in}{0in}{}{}{Figure}{\special{language
"Scientific Word";type "GRAPHIC";display "USEDEF";valid_file "T";width
2.162in;height 1.5056in;depth 0in;original-width 6.9514in;original-height
8.1664in;cropleft "0";croptop "1";cropright "1";cropbottom "0";tempfilename
'SDAC7B02.bmp';tempfile-properties "XPR";}} 
\]

\bigskip

I/O Peripherals:

Input $\rightarrow $Memory$\rightarrow $Processor$\rightarrow $%
(Memory)/Output

\[
\FRAME{itbpF}{2.3774in}{1.6855in}{0in}{}{}{Figure}{\special{language
"Scientific Word";type "GRAPHIC";display "USEDEF";valid_file "T";width
2.3774in;height 1.6855in;depth 0in;original-width 3.1765in;original-height
1.7149in;cropleft "0";croptop "1";cropright "1";cropbottom "0";tempfilename
'SDACA903.bmp';tempfile-properties "XPR";}} 
\]

The intercommunication of all these functional units is carried out with the
help of the \textbf{System Bus}

Full picture:

\[
\FRAME{itbpF}{3.5656in}{2.8798in}{0in}{}{}{Figure}{\special{language
"Scientific Word";type "GRAPHIC";display "USEDEF";valid_file "T";width
3.5656in;height 2.8798in;depth 0in;original-width 11.8531in;original-height
8.6663in;cropleft "0";croptop "1";cropright "1";cropbottom "0";tempfilename
'SDACDE04.bmp';tempfile-properties "XPR";}} 
\]

\bigskip

\subsection{Basics Of Computer Architecture}

It is the \textbf{design of computers}, including their \textbf{instruction
sets, hardware components} and\textbf{\ system organization}.

\textbf{Computer architecture} refers to those attributes of a system
visible to a

programmer or, put another way, those attributes that have a direct impact on

the logical execution of a program.

\bigskip

It has two parts:

1. \textbf{Instruction Set Architecture}(ISA)

\qquad Specification to determine how machine language programs will
interact with the computer

The ISA defines instruction formats, instruction opcodes, registers,
instruction and data memory; the effect of executed instructions on the
registers and memory; and an algorithm for controlling instruction execution.

2. \textbf{Hardware Set Architecture}(HSA)

\qquad Deals with with the major computer subsystems like CPU, Storage, I/O
etc. It includes both the logical design and data flow organization of the
subsystems and hence determines their efficiency.

\bigskip

\[
\FRAME{itbpF}{3.8579in}{1.1606in}{0in}{}{}{Figure}{\special{language
"Scientific Word";type "GRAPHIC";display "USEDEF";valid_file "T";width
3.8579in;height 1.1606in;depth 0in;original-width 16.5387in;original-height
3.9704in;cropleft "0";croptop "1";cropright "1";cropbottom "0";tempfilename
'SDACVL05.bmp';tempfile-properties "XPR";}} 
\]

\bigskip

Illustration:

\bigskip

Suppose we want to add two values: (Example: $2$ and $3$) and store them in
a variable ($z$)

\[
z=2+3 
\]

\bigskip

\end{document}
