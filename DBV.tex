
\documentclass{article}
%%%%%%%%%%%%%%%%%%%%%%%%%%%%%%%%%%%%%%%%%%%%%%%%%%%%%%%%%%%%%%%%%%%%%%%%%%%%%%%%%%%%%%%%%%%%%%%%%%%%%%%%%%%%%%%%%%%%%%%%%%%%%%%%%%%%%%%%%%%%%%%%%%%%%%%%%%%%%%%%%%%%%%%%%%%%%%%%%%%%%%%%%%%%%%%%%%%%%%%%%%%%%%%%%%%%%%%%%%%%%%%%%%%%%%%%%%%%%%%%%%%%%%%%%%%%
%TCIDATA{OutputFilter=LATEX.DLL}
%TCIDATA{Version=5.50.0.2960}
%TCIDATA{<META NAME="SaveForMode" CONTENT="1">}
%TCIDATA{BibliographyScheme=Manual}
%TCIDATA{Created=Thursday, May 09, 2024 22:30:04}
%TCIDATA{LastRevised=Saturday, May 11, 2024 23:10:31}
%TCIDATA{<META NAME="GraphicsSave" CONTENT="32">}
%TCIDATA{<META NAME="DocumentShell" CONTENT="Standard LaTeX\Blank - Standard LaTeX Article">}
%TCIDATA{CSTFile=40 LaTeX article.cst}

\newtheorem{theorem}{Theorem}
\newtheorem{acknowledgement}[theorem]{Acknowledgement}
\newtheorem{algorithm}[theorem]{Algorithm}
\newtheorem{axiom}[theorem]{Axiom}
\newtheorem{case}[theorem]{Case}
\newtheorem{claim}[theorem]{Claim}
\newtheorem{conclusion}[theorem]{Conclusion}
\newtheorem{condition}[theorem]{Condition}
\newtheorem{conjecture}[theorem]{Conjecture}
\newtheorem{corollary}[theorem]{Corollary}
\newtheorem{criterion}[theorem]{Criterion}
\newtheorem{definition}[theorem]{Definition}
\newtheorem{example}[theorem]{Example}
\newtheorem{exercise}[theorem]{Exercise}
\newtheorem{lemma}[theorem]{Lemma}
\newtheorem{notation}[theorem]{Notation}
\newtheorem{problem}[theorem]{Problem}
\newtheorem{proposition}[theorem]{Proposition}
\newtheorem{remark}[theorem]{Remark}
\newtheorem{solution}[theorem]{Solution}
\newtheorem{summary}[theorem]{Summary}
\newenvironment{proof}[1][Proof]{\noindent\textbf{#1.} }{\ \rule{0.5em}{0.5em}}
\input{tcilatex}
\begin{document}

\tableofcontents

\section{Daily Bible Verses}

\subsection{\protect\bigskip May 9, 2024}

\textbf{Mark 9:35}

$35$ And he sat down, and called the twelve, and saith unto them, If any man
desire to be first, the same shall be last of all, and servant of all.

\begin{remark}
Jesus wanted to rectify the distorted concept of being great that his
disciples had in their mind and teach them the values of the kingdom of God
by saying, \textquotedblleft Anyone who wants to be first must be the very
last, and the servant of all.\textquotedblright\ In the kingdom of God, the
greatest are those who are the very last and the servant of all. It requires
a radical change of our value system to accept this teaching of Jesus.
Therefore, if we want to be great in the kingdom of God, we must discard our
worldly concept of greatness, and keep the kindgom concept in our mind and
practice it in our daily lives.
\end{remark}

\bigskip 

\subsection{\protect\bigskip May 10, 2024}

\textbf{Philippians 3:20 KJV}

$20$ For our conversation is in heaven; from whence also we look for the
Saviour, the Lord Jesus Christ:

\begin{remark}
This reminds Us We are Citizens of Heaven. So, all these things that happen
as citizens of a country on earth, but that is temporary citizenship. Our
ultimate citizenship, Philippians 3:20--21 says is in Heaven, where we are
waiting for a savior to come.
\end{remark}

\textbf{Matthew 15:14}

$14$ Let them alone: they be blind leaders of the blind. And if the blind
lead the blind, both shall fall into the ditch.

\begin{remark}
If we follow the teachings of false belief systems, they will only lead us
into the dark pit, as Jesus describes in Matthew 15:14. In this pit, there
is no truth, light, or saving grace. There are only lies, hurt, and
darkness. As believers, we need to be aware of this reality and help point
people to the truth. 
\end{remark}

\bigskip 

\subsection{May 11, 2024}

\textbf{Isaiah 5:8}

$8$ Woe unto them that join house to house, that lay field to field, till
there be no place, that they may be placed alone in the midst of the earth! 

\begin{remark}
Isaiah 5:8--30 contains Isaiah's dire predictions about the upcoming
judgment of Israel. The first "woes" are to the greedy and the
pleasure-seeking drinkers. They will go into exile and to the grave for
refusing to acknowledge God. The Lord then will be exalted for restoring
justice and righteousness.
\end{remark}

\end{document}
