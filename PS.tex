
\documentclass{article}
%%%%%%%%%%%%%%%%%%%%%%%%%%%%%%%%%%%%%%%%%%%%%%%%%%%%%%%%%%%%%%%%%%%%%%%%%%%%%%%%%%%%%%%%%%%%%%%%%%%%%%%%%%%%%%%%%%%%%%%%%%%%%%%%%%%%%%%%%%%%%%%%%%%%%%%%%%%%%%%%%%%%%%%%%%%%%%%%%%%%%%%%%%%%%%%%%%%%%%%%%%%%%%%%%%%%%%%%%%%%%%%%%%%%%%%%%%%%%%%%%%%%%%%%%%%%
\usepackage{amsmath}
\usepackage{amssymb}

\setcounter{MaxMatrixCols}{10}
%TCIDATA{OutputFilter=LATEX.DLL}
%TCIDATA{Version=5.50.0.2960}
%TCIDATA{<META NAME="SaveForMode" CONTENT="1">}
%TCIDATA{BibliographyScheme=Manual}
%TCIDATA{Created=Thursday, May 09, 2024 21:40:36}
%TCIDATA{LastRevised=Friday, May 17, 2024 14:31:35}
%TCIDATA{<META NAME="GraphicsSave" CONTENT="32">}
%TCIDATA{<META NAME="DocumentShell" CONTENT="Standard LaTeX\Blank - Standard LaTeX Article">}
%TCIDATA{CSTFile=40 LaTeX article.cst}

\newtheorem{theorem}{Theorem}
\newtheorem{acknowledgement}[theorem]{Acknowledgement}
\newtheorem{algorithm}[theorem]{Algorithm}
\newtheorem{axiom}[theorem]{Axiom}
\newtheorem{case}[theorem]{Case}
\newtheorem{claim}[theorem]{Claim}
\newtheorem{conclusion}[theorem]{Conclusion}
\newtheorem{condition}[theorem]{Condition}
\newtheorem{conjecture}[theorem]{Conjecture}
\newtheorem{corollary}[theorem]{Corollary}
\newtheorem{criterion}[theorem]{Criterion}
\newtheorem{definition}[theorem]{Definition}
\newtheorem{example}[theorem]{Example}
\newtheorem{exercise}[theorem]{Exercise}
\newtheorem{lemma}[theorem]{Lemma}
\newtheorem{notation}[theorem]{Notation}
\newtheorem{problem}[theorem]{Problem}
\newtheorem{proposition}[theorem]{Proposition}
\newtheorem{remark}[theorem]{Remark}
\newtheorem{solution}[theorem]{Solution}
\newtheorem{summary}[theorem]{Summary}
\newenvironment{proof}[1][Proof]{\noindent\textbf{#1.} }{\ \rule{0.5em}{0.5em}}
\input{tcilatex}
\begin{document}


\section{PS}

\subsection{\protect\bigskip Induction}

\begin{problem}
Let $n$ be a nonnegative integer. Prove that
\end{problem}

\begin{equation*}
\text{ }\underset{L(n)}{\frac{1}{1}-\frac{1}{2}+\frac{1}{3}...-\frac{1}{2n-1}%
+\frac{1}{2n}}=\underset{R(n)}{\frac{1}{n+1}+\frac{1}{n+2}+...+\frac{1}{2n}%
\text{ }}
\end{equation*}

\begin{proof}
We will apply induction on $n$. We

\textbf{Base case}: We must prove $L(0)=R(0)$. This is true, because $L(0)$
and $R(0)$ are empty sums and thus equal to $0.$

\textbf{Induction Step}: Let $m$ be a nonnegative integer. Assume that $%
L(m)=R(m)$(the induction hypothesis). We must prove that $L(m+1)=R(m+1)$. By
definition,

\begin{eqnarray*}
L(m) &=&\frac{1}{1}-\frac{1}{2}+\frac{1}{3}+...+\frac{1}{2m-1}-\frac{1}{2m}
\\
L(m+1) &=&\frac{1}{1}-\frac{1}{2}+\frac{1}{3}+...+\frac{1}{2(m+1)-1}-\frac{1%
}{2(m+1)} \\
&=&\frac{1}{1}-\frac{1}{2}+\frac{1}{3}+...+\frac{1}{2m+1}-\frac{1}{2m+2} \\
&=&(\frac{1}{1}-\frac{1}{2}+\frac{1}{3}+...+\frac{1}{2m-1}-\frac{1}{2m})+%
\frac{1}{2m+1}-\frac{1}{2m+2} \\
&=&L(m)+\frac{1}{2m+1}-\frac{1}{2m+2}
\end{eqnarray*}

\begin{eqnarray*}
R(m) &=&\frac{1}{m+1}+\frac{1}{m+2}+...+\frac{1}{2m} \\
R(m+1) &=&\frac{1}{m+2}+\frac{1}{m+3}+...+\frac{1}{2(m+1)} \\
&=&(\frac{1}{m+1}+\frac{1}{m+2}+...+\frac{1}{2m})-\frac{1}{m+1}+\frac{1}{2m+1%
}+\frac{1}{2m+2} \\
&=&R(m)-\frac{1}{m+1}+\frac{1}{2m+1}+\frac{1}{2m+2}
\end{eqnarray*}

Thus we must show, 
\begin{equation*}
L(m+1)=L(m)+\frac{1}{2m+1}-\frac{1}{2m+2}=\underset{\text{By IH,}L(m)=R(m)}{%
L(m)-\frac{1}{m+1}+\frac{1}{2m+1}+\frac{1}{2m+2}=}R(m)-\frac{1}{m+1}+\frac{1%
}{2m+1}+\frac{1}{2m+2}=R(m+1)
\end{equation*}

as desired. Hence the result follows.

\textbf{This is an identity for finite sums. These types of questions can be
proved in the same.}

\begin{problem}
Let $n$ be a postive integer. A bit shall mean an element of M $\{0,1\}.$An $%
n$-bitstring shall mean an $n$-tuple of bits. For example, $\{1,0,1\}$ is a $%
3$-bitstring. Prove that there exists a list $\{b_{1},b_{2},...,b_{2^{n}}\}$
of all $\ n$-bitstrings exactly once and has the property that for each $%
i\in \{1,2,...,2^{n}\}$, the two $n$-bitstrings $b_{1}$ and $b_{i-1}$ differ
in exactly one entry. [Here $b_{0}$ means $b_{2^{n}}$]
\end{problem}
\end{proof}

\textbf{Rough Sketch}: The case for $n=1$, we have the bitstrings $%
(0)\rightarrow (1)$. For $n=2,$ we have $(0,0)$ , $(0,1)$,$(1,1)$ and $(1,0)$%
.

\begin{equation*}
(0,0)\rightarrow (0,1)\rightarrow (1,1)\rightarrow (1,0)
\end{equation*}

For $n=3$, we first copy the the two bit strings and add $0$ infront of
them, then do same with $1$ in front of them). Then, we merge the two
according to the conditions of the question.

We can use the following algorithm.

\begin{algorithm}
1. Copy the previous $n-1$ bitstrings, append $0$ infront of them and store
them in a list $L$.

2. Copy the previous $n-1$ bitstrings, append $1$ infront of them and store
them in a list $R$.

3. Reverse the order of elements in $R$.

4. Append $R$ to $L$.
\end{algorithm}

\begin{equation*}
\text{1. }(0,0,0)\rightarrow (0,0,1)\rightarrow (0,1,1)\rightarrow (0,1,0)
\end{equation*}%
\begin{equation*}
\text{2. }(1,0,0)\rightarrow (1,0,1)\rightarrow (1,1,1)\rightarrow (1,1,0)
\end{equation*}%
\begin{equation*}
\text{3. }(1,1,0)\rightarrow (1,1,1)\rightarrow (1,0,1)\rightarrow (1,0,0)
\end{equation*}%
\begin{equation*}
\text{4. }(0,0,0)\rightarrow (0,0,1)\rightarrow (0,1,1)\rightarrow
(0,1,0)\rightarrow (1,1,0)\rightarrow (1,1,1)\rightarrow (1,0,1)\rightarrow
(1,0,0)
\end{equation*}

\begin{proof}
We will prove by induction on $n$.

Base case: $n=1$, we have done this above.

Induction Step: Let $m$ be a positive integer. Assume the claim holds for $%
n=m$ (the induction hypothesis). We must prove that the claim holds for $%
n=m+1$. The induction hypothesis tells us that there is a list $%
\{b_{1},b_{2},...,b_{2^{m}}\}$ that satisfies the conditions in the problem.

Now for an $m$ bitstring $b$ let $0b$ be the $(m+1)$ bitstring obtained by
inserting $0$ to the left end of the bitstring.

\begin{equation*}
b=c_{1},c_{2},...,c_{m}\implies 0b=0,c_{1},c_{2},...,c_{m}
\end{equation*}

Similarly let $1b$ denote the $(m+1)$ bitstring obtained by inserting $1$ to
the left end of the bitstring.

\begin{equation*}
b=c_{1},c_{2},...,c_{m}\implies 1b=1,c_{1},c_{2},...,c_{m}
\end{equation*}

\begin{claim}
\begin{equation*}
\{0b_{1},0b_{2},...,0b_{2^{m}},1b_{2^{m}},1b_{2^{m}-1},...,1b_{1}\}
\end{equation*}

forms an $(m+1)$ bitstring that satisfies all the conditions.

\begin{proof}
For $\{0b_{1},0b_{2},...,0b_{2^{m}}\}$ and $\{1b_{1},1b_{2},...,1b_{2^{m}}\}$
it is clear that the given conditions are satisfied. For $%
\{0b_{2^{m}},1b_{2^{m}}\}$, it is clear that the conditions are satisfied
because only one entry is changed. The same reasoning shows that $%
(1b_{1}\rightarrow 0b_{1})$ also satisfies the conditions. Since all the
conditions are satisfied, our claim follows.
\end{proof}
\end{claim}
\end{proof}

\bigskip

\begin{problem}
Prove that any nonempty finite set of integers has a maximum
\end{problem}

Rough Sketch: We restate the problem: Let $n$ be a positive integer, any set
of $n$ integers must have maximum.

\begin{claim}
Let $n$ be a positive integer, any set of $n$ integers must have maximum.

\begin{proof}
We induct on $n.$

Base Case: For $n=1$, there is only one integer in the set and it is the
maximum.

Induction Step: Let $m$ be a postive integer. Assume that any $m$-element
set of integers has a maximum(IH). We must show that any ($m+1)$-element set
of integers has a maximum.

Let $s$ be an arbitrary element of $S$, the ($m+1)$-element set of integers.
Consider $S\backslash \{s\dot{\}}$. We can do this because $S$ is nonempty
by definition.

By the IH, $S\backslash \{s\dot{\}}$ has a maximum. Let $t$ be that maximum.

\begin{claim}
$S$ has a maximum, namely:

\begin{itemize}
\item If $t\geq s$, then $t$ is the maximum.

\item otherwise, s is the maximum.

Thus, our induction step is complete, and our result follows.
\end{itemize}
\end{claim}
\end{proof}
\end{claim}

\begin{problem}
Let $g$ and $h$ be integers s.t. $g\leq h$. Let $b_{g},b_{g+1},...,b_{h}$be
any $h-g+1$ nonzero integers. Assume that $b_{g}\geq 0$. Assume further that

\begin{equation*}
|b_{i+1}-b_{i}|\leq 1...\text{ for every }i\in \{g,g+1,...,h-1\}
\end{equation*}
\end{problem}

Then, $b_{n}>0$ for each n $\in \{g,g+1,...,h\}$.

Rough Sketch: We induct on $n$.

\bigskip

\begin{proof}
We induct on $n$. For $n=g$, $b_{g}\geq 0$ and is nonzero thus $b_{g}>0$.

Let $m$ $\in \{g,g+1,...,h-1\}$. Assume that the claim holds for $%
b_{g},b_{g+1},...,bm$. We must show that the claim holds for $%
b_{g},b_{g+1},...,b_{m+1}$.

Our IH, tells us that $b_{m}>0(b_{m}\geq 1)$, and we have $%
|b_{m+1}-b_{m}|\leq 1$. This implies $b_{m+1}\geq b_{m}-1\geq 0$. Since $%
b_{m+1}$ is nonnegative and nonzero, it must be positive. Hence, our result
follows and our induction is complete.
\end{proof}

\begin{problem}
\bigskip Prove that every integer $n\geq 0$ satisfies

\begin{equation*}
f_{1}+f_{2}+...+f_{n}=f_{n+2}-1
\end{equation*}

\begin{proof}
We induct on $n$.

Base Case: For $n=1$, we have 
\begin{equation*}
f_{1}=f_{3}-1=f_{2}+f_{1}-1=2f_{1}-1\rightarrow f_{1}=1\text{, as desired}
\end{equation*}

Inductive Step: Let $m$ be a postive integer. Assume that $%
f_{1}+f_{2}+...+f_{m}=f_{m+2}-1$(IH). We must show that $%
f_{1}+f_{2}+...+f_{m+1}=f_{m+3}-1$ .

\begin{equation*}
f_{1}+f_{2}+...f_{m}+f_{m+1}=(f_{1}+f_{2}+...f_{m})+f_{m+1}\underset{IH}{=}%
(f_{m+2}-1)+f_{m+1}=f_{m+3}-1
\end{equation*}

Hence, our induction step is complete and our result follows.
\end{proof}
\end{problem}

\begin{problem}
Prove that every integer $n>0$ satisfies

\begin{equation*}
f_{n+1}f_{n-1}-f_{n}^{2}=(-1)^{n}
\end{equation*}
\end{problem}

\bigskip \textbf{Trick}: To rewrite%
\begin{equation*}
f_{n+m+2}
\end{equation*}

as 
\begin{equation*}
f_{n+(m+1)+1}
\end{equation*}

instead.

\bigskip 

\bigskip 

\subsection{Pigeonhole Principle}

\textbf{For Injections}

Let $U$ and $V$ be finite sets such that $|U|$ $>|V|$. Let  $f:U\rightarrow V
$ be any map. Then $f$ cannot be injective.

\bigskip 

\textbf{For "Multi Injections"}

Let $U$ and $V$ be finite sets and $k\in N$. Let  $f:U\rightarrow V$ be any
map. Assume that for each $v\in V$ and $S=\{u$ $|$ $[u\in U]$ $\wedge $\ \ $%
f(u)=v\}$.%
\begin{equation*}
|S|\text{ }\leq k
\end{equation*}

Then $U\leq k|V|$.

\bigskip 

The contrapositive is the pigeonhole principle. Of course, the case for
equality is left for the reader to prove.

\bigskip 

\textbf{For Surjections}

Let $U$ and $V$ be finite sets such that $|U|$ $<|V|$. Let $f:U\rightarrow V$
be any map. Then  $f$ $\ $cannot be surjective.

\end{document}
