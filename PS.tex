
\documentclass{article}
\usepackage{amsmath}

%%%%%%%%%%%%%%%%%%%%%%%%%%%%%%%%%%%%%%%%%%%%%%%%%%%%%%%%%%%%%%%%%%%%%%%%%%%%%%%%%%%%%%%%%%%%%%%%%%%%%%%%%%%%%%%%%%%%%%%%%%%%%%%%%%%%%%%%%%%%%%%%%%%%%%%%%%%%%%%%%%%%%%%%%%%%%%%%%%%%%%%%%%%%%%%%%%%%%%%%%%%%%%%%%%%%%%%%%%%%%%%%%%%%
\usepackage{amssymb}

%TCIDATA{OutputFilter=LATEX.DLL}
%TCIDATA{Version=5.50.0.2960}
%TCIDATA{<META NAME="SaveForMode" CONTENT="1">}
%TCIDATA{BibliographyScheme=Manual}
%TCIDATA{Created=Thursday, May 09, 2024 21:40:36}
%TCIDATA{LastRevised=Thursday, May 09, 2024 22:43:29}
%TCIDATA{<META NAME="GraphicsSave" CONTENT="32">}
%TCIDATA{<META NAME="DocumentShell" CONTENT="Standard LaTeX\Blank - Standard LaTeX Article">}
%TCIDATA{CSTFile=40 LaTeX article.cst}

\newtheorem{theorem}{Theorem}
\newtheorem{acknowledgement}[theorem]{Acknowledgement}
\newtheorem{algorithm}[theorem]{Algorithm}
\newtheorem{axiom}[theorem]{Axiom}
\newtheorem{case}[theorem]{Case}
\newtheorem{claim}[theorem]{Claim}
\newtheorem{conclusion}[theorem]{Conclusion}
\newtheorem{condition}[theorem]{Condition}
\newtheorem{conjecture}[theorem]{Conjecture}
\newtheorem{corollary}[theorem]{Corollary}
\newtheorem{criterion}[theorem]{Criterion}
\newtheorem{definition}[theorem]{Definition}
\newtheorem{example}[theorem]{Example}
\newtheorem{exercise}[theorem]{Exercise}
\newtheorem{lemma}[theorem]{Lemma}
\newtheorem{notation}[theorem]{Notation}
\newtheorem{problem}[theorem]{Problem}
\newtheorem{proposition}[theorem]{Proposition}
\newtheorem{remark}[theorem]{Remark}
\newtheorem{solution}[theorem]{Solution}
\newtheorem{summary}[theorem]{Summary}
\newenvironment{proof}[1][Proof]{\noindent\textbf{#1.} }{\ \rule{0.5em}{0.5em}}
\input{tcilatex}

\begin{document}


\section{PS}

\subsection{\protect\bigskip Induction}

\begin{problem}
Let $n$ be a nonnegative integer. Prove that
\end{problem}

\[
\text{ }\underset{L(n)}{\frac{1}{1}-\frac{1}{2}+\frac{1}{3}...-\frac{1}{2n-1}%
+\frac{1}{2n}}=\underset{R(n)}{\frac{1}{n+1}+\frac{1}{n+2}+...+\frac{1}{2n}%
\text{ }}
\]

\begin{proof}
We will apply induction on $n$. We

\textbf{Base case}: We must prove $L(0)=R(0)$. This is true, because $L(0)$
and $R(0)$ are empty sums and thus equal to $0.$

\textbf{Induction Step}: Let $m$ be a nonnegative integer. Assume that $%
L(m)=R(m)$(the induction hypothesis). We must prove that $L(m+1)=R(m+1)$. By
definition,

\begin{eqnarray*}
L(m) &=&\frac{1}{1}-\frac{1}{2}+\frac{1}{3}+...+\frac{1}{2m-1}-\frac{1}{2m}
\\
L(m+1) &=&\frac{1}{1}-\frac{1}{2}+\frac{1}{3}+...+\frac{1}{2(m+1)-1}-\frac{1%
}{2(m+1)} \\
&=&\frac{1}{1}-\frac{1}{2}+\frac{1}{3}+...+\frac{1}{2m+1}-\frac{1}{2m+2} \\
&=&(\frac{1}{1}-\frac{1}{2}+\frac{1}{3}+...+\frac{1}{2m-1}-\frac{1}{2m})+%
\frac{1}{2m+1}-\frac{1}{2m+2} \\
&=&L(m)+\frac{1}{2m+1}-\frac{1}{2m+2}
\end{eqnarray*}

\begin{eqnarray*}
R(m) &=&\frac{1}{m+1}+\frac{1}{m+2}+...+\frac{1}{2m} \\
R(m+1) &=&\frac{1}{m+2}+\frac{1}{m+3}+...+\frac{1}{2(m+1)} \\
&=&(\frac{1}{m+1}+\frac{1}{m+2}+...+\frac{1}{2m})-\frac{1}{m+1}+\frac{1}{2m+1%
}+\frac{1}{2m+2} \\
&=&R(m)-\frac{1}{m+1}+\frac{1}{2m+1}+\frac{1}{2m+2}
\end{eqnarray*}

Thus we must show, 
\[
L(m+1)=L(m)+\frac{1}{2m+1}-\frac{1}{2m+2}=\underset{\text{By IH,}L(m)=R(m)}{%
L(m)-\frac{1}{m+1}+\frac{1}{2m+1}+\frac{1}{2m+2}=}R(m)-\frac{1}{m+1}+\frac{1%
}{2m+1}+\frac{1}{2m+2}=R(m+1)
\]

as desired. Hence the result follows.

\textbf{This is an identity for finite sums. These types of questions can be
proved in the same.}

\begin{problem}
Let $n$ be a postive integer. A bit shall mean an element of M $\{0,1\}.$An $%
n$-bitstring shall mean an $n$-tuple of bits. For example, $\{1,0,1\}$ is a $%
3$-bitstring. Prove that there exists a list $\{b_{1},b_{2},...,b_{2^{n}}\}$
of all $\ n$-bitstrings exactly once and has the property that for each $%
i\in \{1,2,...,2^{n}\}$, the two $n$-bitstrings $b_{1}$ and $b_{i-1}$ differ
in exactly one entry. [Here $b_{0}$ means $b_{2^{n}}$]

\begin{proof}
\end{proof}
\end{problem}
\end{proof}

\end{document}
