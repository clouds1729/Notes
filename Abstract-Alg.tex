\documentclass[12pt]{article}
\usepackage{amsmath, amsthm, amsfonts, amssymb}
\usepackage{enumerate}
\usepackage{graphicx}
\usepackage{mdframed}
\usepackage{multicol}
\usepackage{verbatim}
\usepackage{tikz}
\usepackage[margin=.8in]{geometry}
\usepackage{fancyhdr}
\usepackage{hyperref}
% load the CM symbol font
\DeclareSymbolFont{arrows}{OMS}{cmsy}{m}{n}
% change the arrows to be taken from the CM symbol font
\DeclareMathSymbol{\leftrightarrow}{\mathrel}{arrows}{"24}
\DeclareMathSymbol{\leftarrow}{\mathrel}{arrows}{"20}
   \let\gets=\leftarrow
\DeclareMathSymbol{\rightarrow}{\mathrel}{arrows}{"21}
   \let\to=\rightarrow
\DeclareMathSymbol{\mapstochar}{\mathrel}{arrows}{"37}
% the bar for making longer arrows
\DeclareMathSymbol{\relbardash}{\mathbin}{arrows}{"00}
\DeclareRobustCommand\relbar{%
  \mathrel{\smash\relbardash}% \smash, because - has the same height as +
}
\geometry{letterpaper}
\linespread{1.2}


\usepackage{xcolor}

\pagestyle{fancy}
\cfoot{Main References:  Algebra, Serg Lang; Abstract Algebra, Dummit and Foote}


\newcommand{\mybox}[2][black]{\colorbox{#1}{#2}}
\newcommand{\mmbox}[2]{\begin{mdframed}\emph{Note:} #1 \end{mdframed}}
\newcommand{\rightarrowline}{%
    \begin{tikzpicture}[baseline=-0.5ex, line width=0.5pt]
        \draw[->] (0,0) -- (0.8,0);
        \draw (0.8,0) -- (0.8,0.15);
    \end{tikzpicture}%
}

\newcommand{\RR}{\mathbb{R}}
\newcommand{\NN}{\mathbb{N}}
\newcommand{\ZZ}{\mathbb{Z}}
\newcommand{\QQ}{\mathbb{Q}}
\newcommand{\f}{\mathbb{f}}

\newcommand{\set}[1]{\left\lbrace #1 \right\rbrace}
\newcommand{\abs}[1]{\left| #1 \right|}
\newcommand{\parens}[1]{\left( #1 \right)}
\newcommand{\brac}[1]{\left[ #1 \right]}
\newcommand{\sol}[1]{\begin{mdframed}\emph{Solution.} #1\end{mdframed}}
\newcommand{\solproof}[1]{\begin{mdframed}\begin{proof} #1\end{proof}\end{mdframed}}
\newcommand{\dis}[1]{\begin{mdframed}\emph{Disclaimer.} #1\end{mdframed}}
\newtheorem{theorem}{Theorem}
\newtheorem{definition}{Definition}
\newtheorem{lemma}{Lemma}
\newtheorem{proposition}{Proposition}


\begin{document}
\thispagestyle{empty}
\color{blue}
\tableofcontents

\dis{I assumed the reader is familiar with a lot of the basics in these notes}
\color{black}
\clearpage
\pagestyle{fancy}

\noindent George K. \hfill \today

\begin{center}
{\large Abstract Algebra}
\end{center}

\section{Basics}
\subsection{Set Theory}

\textbf{Function:} Denoted by $f: A \rightarrow B$ for sets $A$ and $B$ or as $f: a \mapsto b$ for $a \in A,b \in B. $ \\
\textbf{Domain:}  The set $A$ \\
\textbf{Codomain:}  The set $B$\\
\textbf{Range/Image:} $C = [b \in B | b = f(a), a \in A]$ where $C \subseteq B$. $ C = f(A)$\\
\textbf{Preimage/inverse} $D = [a \in A | f(a) \in C]$ where $D \subseteq A$. $ D = f^{-1} (C)$\\
\textbf{Composition} $(g \circ f)(a) = g(f(a)) | f: A \to B, g: B \to C $

\subsubsection{Properties of relations}
\textbf{Injective:} $a_1 \neq a_2, \to f(a_1) \neq f(a_2)$ \\
\textbf{Surjective:} $\forall b \in B, \exists a \in A$ s.t. $f(a) = b $ \\
\textbf{Bijective:} Both injective and surjective.\\
\textbf{Left Inverse:} $(g: B \to A)$ s.t. $ (g \circ f: A \to A)$ is the identity map on A.\\
\textbf{Right Inverse:} $(h: B \to A)$ s.t. $ (f \circ h: B \to B)$ is the identity map on B.\\

\begin{proposition}
\item[1] $f$ is injective iff $f$ has a left inverse
\item[2] $f$ is surjective iff $f$ has a right inverse
\item[3] $f$ is bijective iff $f$ has both a left and right inverse
\end{proposition}

\begin{proof}
\item[1] $\quad f$ is injective $\iff$ for all $a, b \in A$, if $f(a) = f(b)$, then $a = b$ $\iff$ there exists a function $g: B \to A$ such that $g \circ f = \text{id}_A$ $\iff$ $f$ has a left inverse.
\item[2] $\quad f$ is surjective $\iff$ for every $b \in B$, there exists an $a \in A$ such that $f(a) = b$ $\iff$ there exists a function $h: B \to A$ such that $f \circ h = \text{id}_B$ $\iff$ $f$ has a right inverse.
\item[3] $\quad f$ is bijective $\iff$ $f$ is injective and surjective $\iff$ $f$ has a left and right inverse.

\end{proof}

In \textbf{Prop 3}, the left inverse = right inverse = inverse. The inverse is \textbf{unique}.

\begin{definition}
A permutation of a set A is a bijection from A onto itself
\end{definition}

\begin{definition}
If $A \subseteq B$ and $f: B \to C$, we denote the \textbf{restriction} of $f$ to A as $f|_{A}$ 
\end{definition}

\begin{definition}
If $A \subseteq B$ and $f: A \to C$ then there is a function $g: B \to C$ s.t  $g|_{A} = f$. We say $g$ is an \textbf{extension} of $f$ to $B$. 
\end{definition}


\section{Monoids}
\begin{definition}
   
Let $S$ be a set. A mapping
 $S \times S \rightarrow S$ 
is called a $\textbf{law of composition}$ (of $S$ into itself).


\end{definition}

\subsection{Properties of Monoids}
Let $S$ be a set with a law of composition. If $x,y,z \in S$ then we may form their product in two ways: $(xy)z$ and $x(yz)$. If both are equal ($\forall x,y,z$) then we say that the law of composition is \mybox[yellow]{associative}.

An element $e \in S$ such that $e \cdot x = x = x \cdot e$ $\forall x \in S$ is called a \mybox[yellow]{unit/identity element}. When the law of composition is written additively, the identity element is denoted by $0$ and is called the \mybox[yellow]{zero element}.

\begin{proposition}
A unit element is unique.
\end{proposition}
\begin{proof}
Suppose $e'$ is another unit element. Then, we have $e = e \cdot e' = e'$ as desired.
\end{proof}
\mybox[yellow]{Cf. [Algebra, Lang] P3}

A \textbf{monoid} is a set $G$, with a law of composition which is associative, and having an identity/unit element (so in particular $G$ is non-empty).

\mybox[green]{For conciseness, we will assume the reader is comfortable with making simple inductive arguments.}

\subsubsection{Product of Elements in a Monoid}
Let $G$ be a monoid. We define their product inductively. We also define $\prod_{i=1}^{0} x_{i} = e$ i.e., the empty product is equal to the unit/identity element.

\begin{proposition}
Parentheses can be inserted in any manner in the product without changing its value. 
\end{proposition}

\begin{proof}
Induction.
\end{proof}



\subsection{Commutativity in Monoids}
\begin{definition}
Let $S$ and $T$ be sets. $f: S \times S \rightarrow T$. \mybox[yellow]{Commutativity} means $f(x,y) = f(y,x)$ or $xy = yx$.
\end{definition}

\begin{definition}
If the law of composition of $G$ is commutative, we can also call $G$ commutative or abelian.
\end{definition}



\begin{proposition}
Let $G$ be a commutative monoid and $x_i \ldots x_n$ elements of $G$. Let $\phi$ be a bijection of the set of integers $(1, \ldots ,n)$ onto itself. Then 

\[\prod_{i = 1}^{n} x_{\phi(i)} = \prod_{i = 1}^{n}x_i\]
\end{proposition}
\begin{proof}
We prove by induction. For n = 1, the case is obvious. Assume it for n - 1. Let k be an integer such that $\phi(k)=n$. Then
\[  \prod_{i = 1}^{n} x_{\phi(i)} = \prod_{i = 1}^{k-1} x_{\phi(i)} \cdot x_\phi(k) \cdot \prod_{i = 1}^{n-k} x_{\phi(k+i)} = \prod_{i = 1}^{k-1} x_{\phi(i)} \cdot \prod_{i = 1}^{n-k} x_{\phi(k+i)} \cdot x_\phi(k) \]
Define a map $\rho$ of $(1, \ldots, n-1) \text{ into itself by the rule s.t.}$

\[ \rho(c) = \phi(c) \text{  }  [c < k]  \text{ and } \rho(c) = \phi(c+1) \text{  } [c \geq k]\]

\[\prod_{i = 1}^{n} x_{\rho(i)} = \prod_{i = 1}^{k-1} x_{\rho(i)} \cdot \prod_{i = 1}^{n-k} x_{\rho(v+k-1)} \cdot  x_{n} \]

\[\prod_{i = 1}^{n} x_{\rho(i)} = \prod_{i = 1}^{n-1} x_{\rho(i)} \cdot  x_{n} \]


\mybox[yellow]{Cf. [Algebra, Lang] P5} 
\end{proof}


\indent  \indent For $x \in G$, we define $x^{n}$ as 

\[ \prod_{1}^{n} x\]

\subsection{Submonoids}

\begin{definition}
$H \subseteq G$, is a submonoid if $e \in H \text{ and }\forall x, y \in H, x \cdot y \in H$.
\end{definition}
Thus, H is \textbf{closed} under the law of composition. It follows that H is also a monoid.

Example : If $x \in G$ then the subset of powers of $x [x^{n}, n = 0,1, \ldots ]$ is a submonoid of G.\\
The set of integers $\geq 0$ under addition is a monoid(\textcolor{blue}{an additive monoid})

\section{Groups}



\end{document}
