\documentclass[12pt]{article}
\usepackage{amsmath, amsthm, amsfonts, amssymb}
\usepackage{enumerate}
\usepackage{graphicx}
\usepackage{mdframed}
\usepackage{multicol}
\usepackage{verbatim}
\usepackage{tikz}
\usepackage[margin=.8in]{geometry}
\usepackage{fancyhdr}
\usepackage{hyperref}
\geometry{letterpaper}
\linespread{1.2}


\usepackage{xcolor}

\pagestyle{fancy}
\cfoot{Main References:  Algebra, Serg Lang; Abstract Algebra, Dummit and Foote}


\newcommand{\mybox}[2][black]{\colorbox{#1}{#2}}
\newcommand{\mmbox}[2]{\begin{mdframed}\emph{Note:} #1 \end{mdframed}}


\newcommand{\RR}{\mathbb{R}}
\newcommand{\NN}{\mathbb{N}}
\newcommand{\ZZ}{\mathbb{Z}}
\newcommand{\QQ}{\mathbb{Q}}
\newcommand{\f}{\mathbb{f}}

\newcommand{\set}[1]{\left\lbrace #1 \right\rbrace}
\newcommand{\abs}[1]{\left| #1 \right|}
\newcommand{\parens}[1]{\left( #1 \right)}
\newcommand{\brac}[1]{\left[ #1 \right]}
\newcommand{\sol}[1]{\begin{mdframed}\emph{Solution.} #1\end{mdframed}}
\newcommand{\solproof}[1]{\begin{mdframed}\begin{proof} #1\end{proof}\end{mdframed}}

\newtheorem{theorem}{Theorem}
\newtheorem{definition}{Definition}
\newtheorem{lemma}{Lemma}
\newtheorem{proposition}{Proposition}


\begin{document}
\thispagestyle{empty}
\tableofcontents
\clearpage
\pagestyle{fancy}

\noindent George K. \hfill \today

\begin{center}
{\large Abstract Algebra}
\end{center}

\section{Monoids}
\begin{definition}
   
Let $S$ be a set. A mapping
 $S \times S \rightarrow S$ 
is called a $\textbf{law of composition}$ (of $S$ into itself).


\end{definition}

\subsection{Properties of Monoids}
Let $S$ be a set with a law of composition. If $x,y,z \in S$ then we may form their product in two ways: $(xy)z$ and $x(yz)$. If both are equal ($\forall x,y,z$) then we say that the law of composition is \mybox[yellow]{associative}.

An element $e \in S$ such that $e \cdot x = x = x \cdot e$ $\forall x \in S$ is called a \mybox[yellow]{unit/identity element}. When the law of composition is written additively, the identity element is denoted by $0$ and is called the \mybox[yellow]{zero element}.

\begin{proposition}
A unit element is unique.
\end{proposition}
\begin{proof}
Suppose $e'$ is another unit element. Then, we have $e = e \cdot e' = e'$ as desired.
\end{proof}
\mybox[yellow]{Cf. [Algebra, Lang] P3}

A \textbf{monoid} is a set $G$, with a law of composition which is associative, and having an identity/unit element (so in particular $G$ is non-empty).

\mybox[green]{For conciseness, we will assume the reader is comfortable with making simple inductive arguments.}

\subsubsection{Product of Elements in a Monoid}
Let $G$ be a monoid. We define their product inductively. We also define $\prod_{i=1}^{0} x_{i} = e$ i.e., the empty product is equal to the unit/identity element.

\begin{proposition}
Parentheses can be inserted in any manner in the product without changing its value. 
\end{proposition}

\begin{proof}
\textcolor{blue}{Induction. Left as an exercise for the reader.}
\end{proof}



\subsection{Commutativity in Monoids}
\begin{definition}
Let $S$ and $T$ be sets. $f: S \times S \rightarrow T$. \mybox[yellow]{Commutativity} means $f(x,y) = f(y,x)$ or $xy = yx$.
\end{definition}

\begin{definition}
If the law of composition of $G$ is commutative, we can also call $G$ commutative or abelian.
\end{definition}



\begin{proposition}
Let $G$ be a commutative monoid and $x_i \ldots x_n$ elements of $G$. Let $\phi$ be a bijection of the set of integers $(1, \ldots ,n)$ onto itself. Then 

\[\prod_{i = 1}^{n} x_{\phi(i)} = \prod_{i = 1}^{n}x_i\]
\end{proposition}
\begin{proof}
We prove by induction. For n = 1, the case is obvious. Assume it for n - 1. Let k be an integer such that $\phi(k)=n$. Then
\[  \prod_{i = 1}^{n} x_{\phi(i)} = \prod_{i = 1}^{k-1} x_{\phi(i)} \cdot x_\phi(k) \cdot \prod_{i = 1}^{n-k} x_{\phi(k+i)} = \prod_{i = 1}^{k-1} x_{\phi(i)} \cdot \prod_{i = 1}^{n-k} x_{\phi(k+i)} \cdot x_\phi(k) \]
Define a map $\rho$ of $(1, \ldots, n-1) \text{ into itself by the rule s.t.}$

\[ \rho(c) = \phi(c) \text{  }  [c < k]  \text{ and } \rho(c) = \phi(c+1) \text{  } [c \geq k]\]

\[\prod_{i = 1}^{n} x_{\rho(i)} = \prod_{i = 1}^{k-1} x_{\rho(i)} \cdot \prod_{i = 1}^{n-k} x_{\rho(v+k-1)} \cdot  x_{n} \]

\[\prod_{i = 1}^{n} x_{\rho(i)} = \prod_{i = 1}^{n-1} x_{\rho(i)} \cdot  x_{n} \]


\mybox[yellow]{Cf. [Algebra, Lang] P5}

\end{proof}

\end{document}
