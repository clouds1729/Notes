
\documentclass{article}
%%%%%%%%%%%%%%%%%%%%%%%%%%%%%%%%%%%%%%%%%%%%%%%%%%%%%%%%%%%%%%%%%%%%%%%%%%%%%%%%%%%%%%%%%%%%%%%%%%%%%%%%%%%%%%%%%%%%%%%%%%%%%%%%%%%%%%%%%%%%%%%%%%%%%%%%%%%%%%%%%%%%%%%%%%%%%%%%%%%%%%%%%%%%%%%%%%%%%%%%%%%%%%%%%%%%%%%%%%%%%%%%%%%%%%%%%%%%%%%%%%%%%%%%%%%%
%TCIDATA{OutputFilter=LATEX.DLL}
%TCIDATA{Version=5.50.0.2960}
%TCIDATA{<META NAME="SaveForMode" CONTENT="1">}
%TCIDATA{BibliographyScheme=Manual}
%TCIDATA{Created=Thursday, May 09, 2024 22:30:04}
%TCIDATA{LastRevised=Friday, May 17, 2024 11:48:54}
%TCIDATA{<META NAME="GraphicsSave" CONTENT="32">}
%TCIDATA{<META NAME="DocumentShell" CONTENT="Standard LaTeX\Blank - Standard LaTeX Article">}
%TCIDATA{CSTFile=40 LaTeX article.cst}

\newtheorem{theorem}{Theorem}
\newtheorem{acknowledgement}[theorem]{Acknowledgement}
\newtheorem{algorithm}[theorem]{Algorithm}
\newtheorem{axiom}[theorem]{Axiom}
\newtheorem{case}[theorem]{Case}
\newtheorem{claim}[theorem]{Claim}
\newtheorem{conclusion}[theorem]{Conclusion}
\newtheorem{condition}[theorem]{Condition}
\newtheorem{conjecture}[theorem]{Conjecture}
\newtheorem{corollary}[theorem]{Corollary}
\newtheorem{criterion}[theorem]{Criterion}
\newtheorem{definition}[theorem]{Definition}
\newtheorem{example}[theorem]{Example}
\newtheorem{exercise}[theorem]{Exercise}
\newtheorem{lemma}[theorem]{Lemma}
\newtheorem{notation}[theorem]{Notation}
\newtheorem{problem}[theorem]{Problem}
\newtheorem{proposition}[theorem]{Proposition}
\newtheorem{remark}[theorem]{Remark}
\newtheorem{solution}[theorem]{Solution}
\newtheorem{summary}[theorem]{Summary}
\newenvironment{proof}[1][Proof]{\noindent\textbf{#1.} }{\ \rule{0.5em}{0.5em}}
\input{tcilatex}
\begin{document}

\tableofcontents

\section{Daily Bible Verses}

\subsection{\protect\bigskip May 9, 2024}

\textbf{Mark 9:35}

$35$ And he sat down, and called the twelve, and saith unto them, If any man
desire to be first, the same shall be last of all, and servant of all.

\begin{remark}
Jesus wanted to rectify the distorted concept of being great that his
disciples had in their mind and teach them the values of the kingdom of God
by saying, \textquotedblleft Anyone who wants to be first must be the very
last, and the servant of all.\textquotedblright\ In the kingdom of God, the
greatest are those who are the very last and the servant of all. It requires
a radical change of our value system to accept this teaching of Jesus.
Therefore, if we want to be great in the kingdom of God, we must discard our
worldly concept of greatness, and keep the kindgom concept in our mind and
practice it in our daily lives.
\end{remark}

\bigskip

\subsection{\protect\bigskip May 10, 2024}

\textbf{Philippians 3:20 KJV}

$20$ For our conversation is in heaven; from whence also we look for the
Saviour, the Lord Jesus Christ:

\begin{remark}
This reminds Us We are Citizens of Heaven. So, all these things that happen
as citizens of a country on earth, but that is temporary citizenship. Our
ultimate citizenship, Philippians 3:20--21 says is in Heaven, where we are
waiting for a savior to come.
\end{remark}

\textbf{Matthew 15:14}

$14$ Let them alone: they be blind leaders of the blind. And if the blind
lead the blind, both shall fall into the ditch.

\begin{remark}
If we follow the teachings of false belief systems, they will only lead us
into the dark pit, as Jesus describes in Matthew 15:14. In this pit, there
is no truth, light, or saving grace. There are only lies, hurt, and
darkness. As believers, we need to be aware of this reality and help point
people to the truth.
\end{remark}

\bigskip

\subsection{May 11, 2024}

\textbf{Isaiah 5:8}

$8$ Woe unto them that join house to house, that lay field to field, till
there be no place, that they may be placed alone in the midst of the earth!

\begin{remark}
Isaiah 5:8--30 contains Isaiah's dire predictions about the upcoming
judgment of Israel. The first "woes" are to the greedy and the
pleasure-seeking drinkers. They will go into exile and to the grave for
refusing to acknowledge God. The Lord then will be exalted for restoring
justice and righteousness.
\end{remark}

\bigskip

\subsection{May 12, 2024}

\textbf{1 John 5:4 KJV}

$4$ For whatsoever is born of God overcometh the world: and this is the
victory that overcometh the world, even our faith.

\begin{remark}
It is our faith that overcomes the world. But faith must have an object...
and the object of our faith is Jesus Christ. It is our faith in the Person
and Work of the Lord Jesus that gains victory over the world. It is trusting
in the death, burial, and Resurrection of Christ Jesus our Saviour for the
forgiveness of sins and life everlasting, that overcomes the world.
\end{remark}

\textbf{Psalm 19:12-13 KJV}

12 Who can understand his errors? cleanse thou me from secret faults. 13
Keep back thy servant also from presumptuous sins; let them not have
dominion over me: then shall I be upright, and I shall be innocent from the
great transgression.

\begin{remark}
The prior verse referred to "hidden sins," meaning those a person may commit
without realizing it until later. Here, however, David prays the Lord will
restrain him from committing willful sins, called "presumptuous sins" in
this case. Willful sins are committed with the eyes wide open.
\end{remark}

\bigskip

\subsection{\protect\bigskip May 13, 2024}

\textbf{1 Peter 5:7 KJV}

$7$ Casting all your care upon him; for he careth for you.

\begin{remark}
This verse is a reminder to give all your worries and anxieties to God
because He cares for you and is willing to bear your burdens. It is a
message of comfort and assurance of God's love and concern for His people.
\end{remark}

\subsection{May 14, 2024}

\bigskip \textbf{Isaiah 57:1 KJV}

$1$ The righteous perisheth, and no man layeth it to heart: and merciful men
are taken away, none considering that the righteous is taken away from the
evil to come.

\begin{remark}
This verse is a reflection on the fate of the righteous in the face of
death. It suggests that when a righteous person dies, often no one truly
understands or appreciates the significance of their passing. It also
implies that the death of the righteous is often a merciful act of God,
sparing them from future evil or suffering.
\end{remark}

\bigskip

\subsection{May 15, 2024}

\textbf{2 Peter 2:20-21}

$20$ For if after they have escaped the pollutions of the world through the
knowledge of the Lord and Saviour Jesus Christ, they are again entangled
therein, and overcome, the latter end is worse with them than the beginning.
21 For it had been better for them not to have known the way of
righteousness, than, after they have known it, to turn from the holy
commandment delivered unto them.

\begin{remark}
These verses are part of a larger passage in which the apostle Peter is
warning against false teachers and the dangers of falling back into sinful
ways after having known the truth of the gospel. The verses says that those
who turn away from the teachings of Jesus Christ, after having once embraced
them, will face harsher judgment than those who never knew the truth. It
emphasizes the seriousness and the importance of remaining steadfast in
faith.
\end{remark}

\bigskip

\subsection{May 16, 2024}

\textbf{2 Corinthians 4:17}

$17$ For our light affliction, which is but for a moment, worketh for us a
far more exceeding and eternal weight of glory;

\begin{remark}
This verse is part of a larger passage where the apostle Paul is discussing
the trials and tribulations faced by believers. In this verse, Paul
contrasts the temporary nature of the struggles and challenges believers
face ("our light affliction, which is but for a moment") with the eternal
reward and glory that awaits them ("a far more exceeding and eternal weight
of glory"). In a broader biblical context, this verse encourages believers
to persevere through difficulties, understanding that these hardships are
temporary and are ultimately working towards a greater purpose in God's
plan. It speaks to the idea of enduring faith and the hope of a glorious
future with God.
\end{remark}

\bigskip 

\subsection{May 17, 2024}

\textbf{1 Peter 2:16-17}

$16$ As free, and not using your liberty for a cloke of maliciousness, but
as the servants of God. $17$ Honour all men. Love the brotherhood. Fear God.
Honour the king. 

\begin{remark}
This verse urges Christians to live as free people, using their freedom
responsibly and serving God. It emphasizes showing respect to everyone,
loving fellow believers, fearing God, and honoring authority, including the
emperor.
\end{remark}

\end{document}
